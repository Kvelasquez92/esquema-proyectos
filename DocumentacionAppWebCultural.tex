% Generated by Sphinx.
\def\sphinxdocclass{report}
\documentclass[letterpaper,11pt,spanish]{sphinxmanual}
\usepackage{iftex}

\ifPDFTeX
  \usepackage[utf8]{inputenc}
\fi
\ifdefined\DeclareUnicodeCharacter
  \DeclareUnicodeCharacter{00A0}{\nobreakspace}
\fi
\usepackage{cmap}
\usepackage[T1]{fontenc}
\usepackage{amsmath,amssymb,amstext}
\usepackage[spanish]{babel}
\usepackage{times}
\usepackage[Bjarne]{fncychap}
\usepackage{longtable}
\usepackage{sphinx}
\usepackage{multirow}
\usepackage{eqparbox}
\usepackage{float}
\usepackage{titlesec}
 
\titleformat{\chapter}[display]
{\normalfont\huge\bfseries}{}{0pt}{\Huge}
\titlespacing*{\chapter} {0pt}{20pt}{40pt}

\addto\captionsspanish{\renewcommand{\figurename}{Fig.\@ }}
\addto\captionsspanish{\renewcommand{\tablename}{Table }}
\SetupFloatingEnvironment{literal-block}{name=Listing }

\addto\extrasspanish{\def\pageautorefname{page}}

\setcounter{tocdepth}{1}


\title{Documentación de Aplicación Web para Casa de la Cultura Quetzaltenango}
\date{Jul 12, 2016}
\release{0.1}
\author{Kevin Velásquez}
\newcommand{\sphinxlogo}{}
\renewcommand{\releasename}{Release}
\makeindex

\makeatletter
\def\PYG@reset{\let\PYG@it=\relax \let\PYG@bf=\relax%
    \let\PYG@ul=\relax \let\PYG@tc=\relax%
    \let\PYG@bc=\relax \let\PYG@ff=\relax}
\def\PYG@tok#1{\csname PYG@tok@#1\endcsname}
\def\PYG@toks#1+{\ifx\relax#1\empty\else%
    \PYG@tok{#1}\expandafter\PYG@toks\fi}
\def\PYG@do#1{\PYG@bc{\PYG@tc{\PYG@ul{%
    \PYG@it{\PYG@bf{\PYG@ff{#1}}}}}}}
\def\PYG#1#2{\PYG@reset\PYG@toks#1+\relax+\PYG@do{#2}}

\expandafter\def\csname PYG@tok@gd\endcsname{\def\PYG@tc##1{\textcolor[rgb]{0.63,0.00,0.00}{##1}}}
\expandafter\def\csname PYG@tok@gu\endcsname{\let\PYG@bf=\textbf\def\PYG@tc##1{\textcolor[rgb]{0.50,0.00,0.50}{##1}}}
\expandafter\def\csname PYG@tok@gt\endcsname{\def\PYG@tc##1{\textcolor[rgb]{0.00,0.27,0.87}{##1}}}
\expandafter\def\csname PYG@tok@gs\endcsname{\let\PYG@bf=\textbf}
\expandafter\def\csname PYG@tok@gr\endcsname{\def\PYG@tc##1{\textcolor[rgb]{1.00,0.00,0.00}{##1}}}
\expandafter\def\csname PYG@tok@cm\endcsname{\let\PYG@it=\textit\def\PYG@tc##1{\textcolor[rgb]{0.25,0.50,0.56}{##1}}}
\expandafter\def\csname PYG@tok@vg\endcsname{\def\PYG@tc##1{\textcolor[rgb]{0.73,0.38,0.84}{##1}}}
\expandafter\def\csname PYG@tok@vi\endcsname{\def\PYG@tc##1{\textcolor[rgb]{0.73,0.38,0.84}{##1}}}
\expandafter\def\csname PYG@tok@mh\endcsname{\def\PYG@tc##1{\textcolor[rgb]{0.13,0.50,0.31}{##1}}}
\expandafter\def\csname PYG@tok@cs\endcsname{\def\PYG@tc##1{\textcolor[rgb]{0.25,0.50,0.56}{##1}}\def\PYG@bc##1{\setlength{\fboxsep}{0pt}\colorbox[rgb]{1.00,0.94,0.94}{\strut ##1}}}
\expandafter\def\csname PYG@tok@ge\endcsname{\let\PYG@it=\textit}
\expandafter\def\csname PYG@tok@vc\endcsname{\def\PYG@tc##1{\textcolor[rgb]{0.73,0.38,0.84}{##1}}}
\expandafter\def\csname PYG@tok@il\endcsname{\def\PYG@tc##1{\textcolor[rgb]{0.13,0.50,0.31}{##1}}}
\expandafter\def\csname PYG@tok@go\endcsname{\def\PYG@tc##1{\textcolor[rgb]{0.20,0.20,0.20}{##1}}}
\expandafter\def\csname PYG@tok@cp\endcsname{\def\PYG@tc##1{\textcolor[rgb]{0.00,0.44,0.13}{##1}}}
\expandafter\def\csname PYG@tok@gi\endcsname{\def\PYG@tc##1{\textcolor[rgb]{0.00,0.63,0.00}{##1}}}
\expandafter\def\csname PYG@tok@gh\endcsname{\let\PYG@bf=\textbf\def\PYG@tc##1{\textcolor[rgb]{0.00,0.00,0.50}{##1}}}
\expandafter\def\csname PYG@tok@ni\endcsname{\let\PYG@bf=\textbf\def\PYG@tc##1{\textcolor[rgb]{0.84,0.33,0.22}{##1}}}
\expandafter\def\csname PYG@tok@nl\endcsname{\let\PYG@bf=\textbf\def\PYG@tc##1{\textcolor[rgb]{0.00,0.13,0.44}{##1}}}
\expandafter\def\csname PYG@tok@nn\endcsname{\let\PYG@bf=\textbf\def\PYG@tc##1{\textcolor[rgb]{0.05,0.52,0.71}{##1}}}
\expandafter\def\csname PYG@tok@no\endcsname{\def\PYG@tc##1{\textcolor[rgb]{0.38,0.68,0.84}{##1}}}
\expandafter\def\csname PYG@tok@na\endcsname{\def\PYG@tc##1{\textcolor[rgb]{0.25,0.44,0.63}{##1}}}
\expandafter\def\csname PYG@tok@nb\endcsname{\def\PYG@tc##1{\textcolor[rgb]{0.00,0.44,0.13}{##1}}}
\expandafter\def\csname PYG@tok@nc\endcsname{\let\PYG@bf=\textbf\def\PYG@tc##1{\textcolor[rgb]{0.05,0.52,0.71}{##1}}}
\expandafter\def\csname PYG@tok@nd\endcsname{\let\PYG@bf=\textbf\def\PYG@tc##1{\textcolor[rgb]{0.33,0.33,0.33}{##1}}}
\expandafter\def\csname PYG@tok@ne\endcsname{\def\PYG@tc##1{\textcolor[rgb]{0.00,0.44,0.13}{##1}}}
\expandafter\def\csname PYG@tok@nf\endcsname{\def\PYG@tc##1{\textcolor[rgb]{0.02,0.16,0.49}{##1}}}
\expandafter\def\csname PYG@tok@si\endcsname{\let\PYG@it=\textit\def\PYG@tc##1{\textcolor[rgb]{0.44,0.63,0.82}{##1}}}
\expandafter\def\csname PYG@tok@s2\endcsname{\def\PYG@tc##1{\textcolor[rgb]{0.25,0.44,0.63}{##1}}}
\expandafter\def\csname PYG@tok@nt\endcsname{\let\PYG@bf=\textbf\def\PYG@tc##1{\textcolor[rgb]{0.02,0.16,0.45}{##1}}}
\expandafter\def\csname PYG@tok@nv\endcsname{\def\PYG@tc##1{\textcolor[rgb]{0.73,0.38,0.84}{##1}}}
\expandafter\def\csname PYG@tok@s1\endcsname{\def\PYG@tc##1{\textcolor[rgb]{0.25,0.44,0.63}{##1}}}
\expandafter\def\csname PYG@tok@ch\endcsname{\let\PYG@it=\textit\def\PYG@tc##1{\textcolor[rgb]{0.25,0.50,0.56}{##1}}}
\expandafter\def\csname PYG@tok@m\endcsname{\def\PYG@tc##1{\textcolor[rgb]{0.13,0.50,0.31}{##1}}}
\expandafter\def\csname PYG@tok@gp\endcsname{\let\PYG@bf=\textbf\def\PYG@tc##1{\textcolor[rgb]{0.78,0.36,0.04}{##1}}}
\expandafter\def\csname PYG@tok@sh\endcsname{\def\PYG@tc##1{\textcolor[rgb]{0.25,0.44,0.63}{##1}}}
\expandafter\def\csname PYG@tok@ow\endcsname{\let\PYG@bf=\textbf\def\PYG@tc##1{\textcolor[rgb]{0.00,0.44,0.13}{##1}}}
\expandafter\def\csname PYG@tok@sx\endcsname{\def\PYG@tc##1{\textcolor[rgb]{0.78,0.36,0.04}{##1}}}
\expandafter\def\csname PYG@tok@bp\endcsname{\def\PYG@tc##1{\textcolor[rgb]{0.00,0.44,0.13}{##1}}}
\expandafter\def\csname PYG@tok@c1\endcsname{\let\PYG@it=\textit\def\PYG@tc##1{\textcolor[rgb]{0.25,0.50,0.56}{##1}}}
\expandafter\def\csname PYG@tok@o\endcsname{\def\PYG@tc##1{\textcolor[rgb]{0.40,0.40,0.40}{##1}}}
\expandafter\def\csname PYG@tok@kc\endcsname{\let\PYG@bf=\textbf\def\PYG@tc##1{\textcolor[rgb]{0.00,0.44,0.13}{##1}}}
\expandafter\def\csname PYG@tok@c\endcsname{\let\PYG@it=\textit\def\PYG@tc##1{\textcolor[rgb]{0.25,0.50,0.56}{##1}}}
\expandafter\def\csname PYG@tok@mf\endcsname{\def\PYG@tc##1{\textcolor[rgb]{0.13,0.50,0.31}{##1}}}
\expandafter\def\csname PYG@tok@err\endcsname{\def\PYG@bc##1{\setlength{\fboxsep}{0pt}\fcolorbox[rgb]{1.00,0.00,0.00}{1,1,1}{\strut ##1}}}
\expandafter\def\csname PYG@tok@mb\endcsname{\def\PYG@tc##1{\textcolor[rgb]{0.13,0.50,0.31}{##1}}}
\expandafter\def\csname PYG@tok@ss\endcsname{\def\PYG@tc##1{\textcolor[rgb]{0.32,0.47,0.09}{##1}}}
\expandafter\def\csname PYG@tok@sr\endcsname{\def\PYG@tc##1{\textcolor[rgb]{0.14,0.33,0.53}{##1}}}
\expandafter\def\csname PYG@tok@mo\endcsname{\def\PYG@tc##1{\textcolor[rgb]{0.13,0.50,0.31}{##1}}}
\expandafter\def\csname PYG@tok@kd\endcsname{\let\PYG@bf=\textbf\def\PYG@tc##1{\textcolor[rgb]{0.00,0.44,0.13}{##1}}}
\expandafter\def\csname PYG@tok@mi\endcsname{\def\PYG@tc##1{\textcolor[rgb]{0.13,0.50,0.31}{##1}}}
\expandafter\def\csname PYG@tok@kn\endcsname{\let\PYG@bf=\textbf\def\PYG@tc##1{\textcolor[rgb]{0.00,0.44,0.13}{##1}}}
\expandafter\def\csname PYG@tok@cpf\endcsname{\let\PYG@it=\textit\def\PYG@tc##1{\textcolor[rgb]{0.25,0.50,0.56}{##1}}}
\expandafter\def\csname PYG@tok@kr\endcsname{\let\PYG@bf=\textbf\def\PYG@tc##1{\textcolor[rgb]{0.00,0.44,0.13}{##1}}}
\expandafter\def\csname PYG@tok@s\endcsname{\def\PYG@tc##1{\textcolor[rgb]{0.25,0.44,0.63}{##1}}}
\expandafter\def\csname PYG@tok@kp\endcsname{\def\PYG@tc##1{\textcolor[rgb]{0.00,0.44,0.13}{##1}}}
\expandafter\def\csname PYG@tok@w\endcsname{\def\PYG@tc##1{\textcolor[rgb]{0.73,0.73,0.73}{##1}}}
\expandafter\def\csname PYG@tok@kt\endcsname{\def\PYG@tc##1{\textcolor[rgb]{0.56,0.13,0.00}{##1}}}
\expandafter\def\csname PYG@tok@sc\endcsname{\def\PYG@tc##1{\textcolor[rgb]{0.25,0.44,0.63}{##1}}}
\expandafter\def\csname PYG@tok@sb\endcsname{\def\PYG@tc##1{\textcolor[rgb]{0.25,0.44,0.63}{##1}}}
\expandafter\def\csname PYG@tok@k\endcsname{\let\PYG@bf=\textbf\def\PYG@tc##1{\textcolor[rgb]{0.00,0.44,0.13}{##1}}}
\expandafter\def\csname PYG@tok@se\endcsname{\let\PYG@bf=\textbf\def\PYG@tc##1{\textcolor[rgb]{0.25,0.44,0.63}{##1}}}
\expandafter\def\csname PYG@tok@sd\endcsname{\let\PYG@it=\textit\def\PYG@tc##1{\textcolor[rgb]{0.25,0.44,0.63}{##1}}}

\def\PYGZbs{\char`\\}
\def\PYGZus{\char`\_}
\def\PYGZob{\char`\{}
\def\PYGZcb{\char`\}}
\def\PYGZca{\char`\^}
\def\PYGZam{\char`\&}
\def\PYGZlt{\char`\<}
\def\PYGZgt{\char`\>}
\def\PYGZsh{\char`\#}
\def\PYGZpc{\char`\%}
\def\PYGZdl{\char`\$}
\def\PYGZhy{\char`\-}
\def\PYGZsq{\char`\'}
\def\PYGZdq{\char`\"}
\def\PYGZti{\char`\~}
% for compatibility with earlier versions
\def\PYGZat{@}
\def\PYGZlb{[}
\def\PYGZrb{]}
\makeatother

\renewcommand\PYGZsq{\textquotesingle}

\begin{document}

\maketitle
\tableofcontents
\phantomsection\label{index::doc}

Contenido:


\chapter{Aplicación Web ``Solo Cultura y Arte''}
\label{objetivos::doc}\label{objetivos:aplicacion-web-solo-cultura-y-arte}\label{objetivos:bienvenido-a-la-documentacion-de-aplicacion-web-para-casa-de-la-cultura-quetzaltenango}

\section{Breve historia del proyecto}
\label{objetivos:breve-historia-del-proyecto}
El idea de la implementación este sistema nace a partir del interés de \textless{}\textless{}La Casa
la Cultura\textgreater{}\textgreater{} por favorecer la cultura en Quetzaltenando, supliendo varias necesidades
que se identificaron a la hora de realizar sus tareas. Tareas como permitir el acceso a
eventos y artistas de distintos tipos (música, pintura, fotografía, literatura, etc)
al público en general, agilizar la reservación de sitios y la gestión de eventos.
Una de las mayores necesidades que en la institución del cliente se presenta es el poder
llevar de forma óptima y sencilla el registro e información histórica de los eventos y las
personas e instituciones involucradas en ellos.
Por lo que el encargado de gestión de proyectos de La Casa de la Cultura llamado Ernesto Pacheco
se abocó a los participantes del curso de Análisis y Diseño II de la Universidad Rafael Landívar
para formar un acuerdo en el que se pueda implementar un sistema de apoyo para cubrir sus necesidades.

A partir de las necesidades de la institución, la intención de implementar este proyecto es brindar
un fácil acceso a los artistas y personas en general para que puedan contactar a los artistas
por medio de la aplicación y organizar más facilmente eventos culturales.
En concreto se pretende llevar un óptimo control en el registro de artistas y en la coordinación de
eventos, hacer reservaciones de los espacios municipales disponibles para eventos, informar a las
personas con datos importantes sobre cultura y arte y brindar datos estadísticos que puedan ser
importantes para la dirección de La Casa de la Cultura.

La ventaja de tener un sistema como este se verá reflejada en una mayor cantidad de eventos de cultura
en Quetzaltenando y la región al tener una forma fácil y accesible para todo público de contactar a los
artistas y a la misma casa de la cultura para coordinar un evento, dicho de otra forma, incrementará
la cultura en Quetzaltenango.


\section{Casa de la cultura Quetzaltenango}
\label{objetivos:casa-de-la-cultura-quetzaltenango}
La Casa de la Cultura es una institución municipal de Quetzaltenango la cual se
encarga de la coordinación de los eventos culturales del lugar. Su función es
meramente administrativa ya que lleva el control de las instalaciones municipales
disponibles para eventos culturales, tambien se encarga en algunos casos de la
logística y gestión de todas las condiciones necesarias para llevar a cabo un evento.

Para el caso de la creación/gestión de eventos, La Casa de la cultura se encuentra
con tres posibles escenarios:
\begin{itemize}
\item {} 
Producción externa: cuando la persona o institución sólo solicita la reservación de
alguna de las instalaciones disponibles, por cierto tiempo para la realización de su
evento. Entonces La Casa de la cultura solo se encarga de verificar la disponibilidad
del (de las) instalaciones y sus horarios y luego de la reservación del lugar, es
la persona ajena quien se encarga del resto del evento.

\item {} 
Producción interna: cuando es la propia Casa de la Cultura la impulsora del evento,
por ejemplo, alguna actividad como conferencias, celebraciones o conciertos de musica,
entre otros, en donde a La Casa de la cultura le corresponde toda la logística del
evento, desde reservación y acondicionamiento de instalaciones, contratación de artistas,
invitación de las personas y promoción del evento y conseguir cualquier otro recurso necesario.

\item {} 
Producción mixta: u organizacional, en este escenario, La Casa de la Cultura comparte
la creación del evento. Se delegan y distribuyen las tareas anteriormente mencionadas entre
La Casa de la Cultura y la(s) otra(s) instituciones según sea el evento.

\end{itemize}


\section{Misión del proyecto}
\label{objetivos:mision-del-proyecto}
Apoyar a la dirección de La Casa de la Cultura a tomar las decisiones más convenientes
en cuanto a qué metodologías mejorar o replantear en la gestión de eventos o contacto de artistas
con el fin de que sea más productiva. Esto por medio de una plataforma de fácil acceso
al público y artistas para poder obtener y administrar la información de la institución
en forma eficiente y segura, la cual al mismo tiempo debe ser relevante y confiable y así
garantizar buenos resultados en las decisiones que puedan tomar a partir de la información mostrada.


\section{Objetivos del proyecto}
\label{objetivos:objetivos-del-proyecto}\begin{itemize}
\item {} 
Facilitar la creación de un nuevo evento: desplegando información de instalaciones
y artistas disponibles, horarios y eventos proximos de tal manera que esta operación
tome menos tiempo hacerla.

\item {} 
Presentar menús y opciones que permitan consultar información de la programación
de eventos de forma ordenada de tal manera que al crear eventos todos estén bien
distribuidos en el tiempo.

\item {} 
Que se tenga un mapa incrustado en la vista de cada evento, por medio de la aplicacion de
Google Maps, de tal forma que todas las personas interesadas puedan ubicar fácilmente el
lugar del evento.

\item {} 
Ofrecer en cada evento un botón con la opción de sincronizar el evento con el calendario de
Google, esto para que los usuarios no registrados tengan presente los eventos de su interés.

\item {} 
Proporcionar distintas opciones de que tipo de estadísticas se desea consultar, para que
de esta forma se puedan enfocar en el análisis de datos más específicos y puntuales.

\item {} 
En cada evento implementar un área para aceptar comentarios de usuarios registrados y así
conocer multiples opiniones en cuanto al grado de aceptación de los eventos realizados.

\end{itemize}


\chapter{El proyecto}
\label{proyecto::doc}\label{proyecto:el-proyecto}

\section{Descripción General}
\label{proyecto:descripcion-general}
El proyecto consiste en implementar una aplicación web que sea accedida desde
cualquier parte del mundo, la cual deberá contar con un directorio de artistas
clasificados por categoría, según el tipo de arte que practiquen. Deberá permitir
a los usuarios visitantes poder registrarse, por medio de Facebook o por
e-mail en su defecto para tener una mayor interacción en la aplicación, como
calificar eventos y dejar reseñas, seguir artistas y/o eventos, suscribirse
para recibir información de los temas de su interés, etc.

La aplicación debe permitir registrar nuevos artistas, cuya información debe ser
validada por un delegado en el área de comunicación de la casa de la cultura.
Se debe poder agregar notas educativas, referente a cualquier tipo cultura.

Del lado de la casa de la cultura, la aplicación debe permitir calendarizar nuevos
eventos y asignar a estos cualquiera de los artistas que esté disponible, al igual
que la reservación de alguna instalación propia que este disponible o una instalación
externa, con la opción de incluir la ubicación en formato de mapa. También, deben
poder consultar información estadística sobre la cantidad de eventos en un lapso,
usuarios y artistas registrados, categorías preferidas, entre otras.


\section{Modulos}
\label{proyecto:modulos}

\subsection{Artistas:}
\label{proyecto:artistas}
En éste módulo se llevará el control de los artistas que se registrarán en el
sistema. Acá se permitirá gestionar a los artistas que estarán en el directorio
de la aplicación, es decir, operaciones como registrar un artista, editar su propio
perfil, borrar la cuenta, entre otras operaciones. Acá también se abarcará la
validación de los datos del artista que desee registrarse y se proporcionará una
alternativa para que cualquier usuario o artista registrado pueda contactar a otro artista,
bien sea por su dirección de correo u obteniendo su numero de teléfono, de tal manera
que quede registrado en el sistema que el artista fue contactado.

Ver {\hyperref[disenio:etiqueta1]{\crossref{\DUrole{std,std-ref}{Casos de uso módulo de Artistas:}}}}.


\subsection{Usuarios:}
\label{proyecto:usuarios}
En este módulo, de manera similiar al de los artistas, permitirá llevar el control
de los usuarios particulares que deseen registrarse en el sistema. Con la salvedad
de que estos tienen la alternativa de registrarse mediante su cuenta de Facebook,
desde donde se obtendrán sus datos, o en caso contrario ingresará su información
personal necesaria y una dirección de correo electrónico. Este módulo permitirá editar
el perfil de usuario, dejar reseñas sobre eventos eventos, acceso a leer notas educativas
y calificar eventos.

Ver {\hyperref[disenio:etiqueta2]{\crossref{\DUrole{std,std-ref}{Casos de uso módulo de Usuarios:}}}}.


\subsection{Gestión:}
\label{proyecto:gestion}
Este módulo será uno de los más complejos por las operaciones que permitirá realizar.
Este será el módulo que estará enfocado a los eventos, acá se permitirá asignar los
artístas al evento y también se desplegará y agregará toda la información concerniente
al evento, como los horarios disponibles de los artistas e instalaciones disponibles para
los eventos. Acá se hará la programación de eventos o el cronograma, a la cual tendrán
acceso todos los usuarios y artistas, y donde se permitirá darle seguimiento a uno o varios
eventos en particular, por medio de calendar o por notificaciónes de correo electrónico.
Este módulo estará más orientado a los responsables de manejar la aplicación en la casa de la cultura.

Ver {\hyperref[disenio:etiqueta3]{\crossref{\DUrole{std,std-ref}{Casos de uso módulo de Gestión:}}}}.


\subsection{Recolección:}
\label{proyecto:recoleccion}
Este módulo, como su nombre lo indica, será el encargado de la recolección de datos
estadísticos de los demás módulos, fechas, cantidades, categorías, visitas, lugares, entre otros.
Se encargará de consultar datos correspondientes a los demás módulos y luego pasarle
los filtros adecuados para presentar la información estadística relevante a los usuarios
de casa de la cultura.


\subsection{Control:}
\label{proyecto:control}
Este módulo será el encargado de brindar herramientas para administrar la información que se
muestra en la página, desde acá se podrán gestionar los post educativos de libre acceso, que
serán normalmente post sobre cualquier tipo de arte y cultura, es decir, agregar, editar y
elmininar los post que en algún momento se consideren obsoletos y no aplicables, desde acá
existirá también la opción de eliminar cuentas de artistas, artistas que ya no volverán a estar
disponibles o que por politicas de la casa de la cultura ya no puedan estar en el directorio.
De igual forma como con los artistas será para los usuarios registrados. Acá tambíen se permitirá
la gestión de las categorías y en algún momento la reclasificación de eventos a otra categoría o
también eliminar algunos eventos que algún momento queden obsoletos.

Ver {\hyperref[disenio:etiqueta4]{\crossref{\DUrole{std,std-ref}{Casos de uso módulo de Control:}}}}.


\chapter{Diseño de la aplicacíon}
\label{disenio:diseno-de-la-aplicacion}\label{disenio::doc}

\section{Casos de Uso}
\label{disenio:casos-de-uso}
A continuación se presentarán los diagramas de casos de uso para cada módulo. Estos diagramas
sirven para representar de de forma gráfica los distintos actores que tendrá el sistema,
es decir los distintos tipos de usuarios que tendrán interacción con el programa y las distintas
actividadesque cada uno de ellos podrá realizar en el programa, según cada módulo y el nivel de acceso
que tenga cada actor.


\subsection{Casos de uso módulo de Artistas:}
\label{disenio:casos-de-uso-modulo-de-artistas}\label{disenio:etiqueta1}\begin{figure}[H]
\centering

\includegraphics[width=0.750\linewidth]{{moduloArtistas}.png}
\end{figure}

Este módulo tiene los actores Artista y Administrador puesto que estos dos son los que realizarán
las tareas descritas en el diagrama, las cuales estan relacionadas con la gestión de Artistas.


\subsection{Casos de uso módulo de Usuarios:}
\label{disenio:casos-de-uso-modulo-de-usuarios}\label{disenio:etiqueta2}\begin{figure}[H]
\centering

\includegraphics[width=0.750\linewidth]{{moduloUsuarios}.png}
\end{figure}

Para este diagrama serán los visitantes particulares de la página los que podrán realizar dichas tareas
específicamente, tal como se muestra, será un unico actor. En algún momento podría involucrarse acá
la tarea de eliminar el usuario, en donde podría también entrar a participar el actor Administrador.


\subsection{Casos de uso módulo de Gestión:}
\label{disenio:etiqueta3}\label{disenio:casos-de-uso-modulo-de-gestion}\begin{figure}[H]
\centering

\includegraphics[width=0.750\linewidth]{{moduloGestion}.png}
\end{figure}

En esté módulo se incluyen todas las tareas que están relacionadas con la creación y manejo de eventos,
las cuales corresponden a La Casa de la cultura, que en este caso es el usuario Administrador.
Acá también se abarca las opciones de suscripción de los usuarios.


\subsection{Casos de uso módulo de Control:}
\label{disenio:etiqueta4}\label{disenio:casos-de-uso-modulo-de-control}\begin{figure}[H]
\centering

\includegraphics[width=0.750\linewidth]{{moduloControl}.png}
\end{figure}

Este módulo es el que incluirá las tareas más generales del sistema, las cuales están más relacionadas
con el acondicionamiento gráfico de la información dentro del programa, es por eso que solo un usuario
con privilegios de administrador podrá hacerlas, en este caso será La Casa de la Cultura.


\section{Clases}
\label{disenio:clases}
En este tipo de diagrama se describe la idea principal de la lógica de programación del proyecto, normalmente
el diagrama es bastante más extenso, pero para fines practicos se define acá la parte esencial, para comprenderlo
mejor es recomendable consultar también el diagrama de BD ya que tiene bastante relación. Se puede intuir en
principio que el diagrama de clases también necesitará una clase por cada tabla del diagrama de BD, y sus asociaciones
con entre clases serán también basicamente las mismas.
\begin{figure}[H]
\centering

\includegraphics[width=0.750\linewidth]{{diagramaClases}.png}
\end{figure}


\section{Diagrama de BD}
\label{disenio:diagrama-de-bd}
Este diagrama describe la lógica del almacenamiento de los datos en la computadora y la relación que tendrán
estos datos entre sí. Cada tabla representa una entidad y puede entenderse que cada entidad será almacenada
con los campos indicados ya que así esta definido su formato.
\begin{figure}[H]
\centering

\includegraphics[width=0.750\linewidth]{{modeloRelacional}.png}
\end{figure}


\chapter{Desarrollo del proyecto}
\label{tiempos:desarrollo-del-proyecto}\label{tiempos::doc}

\section{Cómo se trabajará}
\label{tiempos:como-se-trabajara}
Explicación de cómo se trabajará la aplicación, como ciclo de vida, prototipos,
etc.


\section{Actividades a realizar}
\label{tiempos:actividades-a-realizar}
Listado de actividades importantes que el cliente debe de tener en cuenta, como por
ejemplo, el análisis, la implementación, capacitaciones, etc.


\section{Entregables}
\label{tiempos:entregables}
Se listan y se describen cada uno de los elementos que se entregarán al final
del proyecto, como código fuente, manuales, etc.


\section{Tabla de tiempos y costos}
\label{tiempos:tabla-de-tiempos-y-costos}
Se crea una tabla de tiempos estipulados de desarrollo al igual que los costos,
pero en el caso del proyecto solamente se colocarán los tiempos, ya que la
evaluación de costos se verá en otros cursos mas adelante.


\section{Condiciones}
\label{tiempos:condiciones}
Condiciones legales, administrativas o de otra índole que el cliente debe de
tomar en cuenta para la realización del proyecto, como fechas de pago, licencias,
requerimientos, etc.



\renewcommand{\indexname}{Index}
\printindex
\end{document}
